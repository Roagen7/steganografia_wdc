\documentclass{article}

\title{Podstawy steganografii i steganoanalizy}
\author{Dominik Lau, Sebastian Kutny, Tomasz Lewandowski, Maciej Krzyżanowski}

\usepackage{blindtext}
\usepackage{amsmath}
\usepackage[utf8]{inputenc}
\usepackage[polish]{babel}
\usepackage[T1]{fontenc}
\usepackage{listings}
\usepackage{color}
\usepackage{amssymb}
\usepackage{esvect}
\usepackage{graphicx}
\usepackage{hyperref}
\usepackage{float}


\graphicspath{ {./obrazy/} }

\definecolor{dkgreen}{rgb}{0,0.6,0}
\definecolor{gray}{rgb}{0.5,0.5,0.5}
\definecolor{mauve}{rgb}{0.58,0,0.82}

\lstset{frame=tb,
  language=Python,
  aboveskip=3mm,
  belowskip=3mm,
  showstringspaces=false,
  columns=flexible,
  basicstyle={\small\ttfamily},
  numbers=none,
  numberstyle=\tiny\color{gray},
  keywordstyle=\color{blue},
  commentstyle=\color{dkgreen},
  stringstyle=\color{mauve},
  breaklines=true,
  breakatwhitespace=true,
  tabsize=3
}


\begin{document}

\maketitle

\tableofcontents

\section{Czym jest steganografia? Do czego służy?}
Steganografia polega na ukrywaniu informacji przez ukrywanie komunikacji w innej formie transmisji danych
np. w obrazkach,  plikach dźwiękowych, tekstowych.  Zastosowania steganografii
\begin{itemize}
	\item omijanie cenzury/szpiegostwo
	\item umieszczanie znaków wodnych
	\item ukryta wymiana danych
	\item dodawanie metadanych do plików (np. znaki sterujące)
	\item numery seryjne drukarek (za pomocą małych kropek)
	\item wprowadzanie opóźnień w pakietach sieciowych
	\item zastosowania w VoIP (steganofonia)
	\item zabezpieczanie banknotów (np. EURion constellation)
\end{itemize}
\begin{figure}
	\centering
	\includegraphics[width=5cm]{stego_drukarkowa}
	\caption{"kropki" zamieszczane przez drukarki}
\end{figure}
Steganografia może zatem realizować następujące funkcje bezpieczeństwa
\begin{itemize}
	\item poufność
	\item autentyczność
	\item niezaprzeczalność
	\item integralność
\end{itemize}
Porównanie kryptografii i steganografii
\begin{center}
\begin{tabular}{c | c  c }
& kryptografia & steganografia \\
\hline
cel & zapewnienie poufności & ukrycie komunikacji \\
obecność klucza & tak & opcjonalna \\
widoczność danych & nie & tak \\
modyfikacja struktury  \\
 przetwarzanych danych & nie & tak
\end{tabular}
\end{center}
\subsection{Słowniczek}
\begin{itemize}
	\item stegosystem - połączenie metod i narzędzi służących do tworzenia ukrytego kanału do przekazywania informacji
	\item wiadomość (payload) - przesyłane dane
	\item kontener/nośnik (carrier) - to wszelkie dane służące do ukrycia tajnej wiadomości
	\item stegokontener - dane i ukryta w nich tajna wiadomość
	\item kanał steganograficzny (stegochannel) - kanał transmisji stegokontenera
	\item klucz (stegokey) - tajny klucz potrzebny do ukrycia stegokontenera
\end{itemize}
\begin{figure}
	\centering
	\includegraphics[width=8cm]{model_steganografii}
	\caption{model steganografii}
\end{figure}
\subsection{Podział steganografii}
Ze względu na kontener
\begin{itemize}
	\item w plikach tekstowych
	\item w plikach audio
	\item w obrazach
	\item w ramkach różnych protokołów
	\item w plikach wykonawczych
	\item inne...
\end{itemize}
Ze względu na metodę modyfikacji nośnika
\begin{itemize}
	\item \textbf{metody substytucji} - zamiana nadmiarowych danych nośnika
	\item \textbf{metody transformacyjne} - modyfikacja postaci falowej nośnika
	\item metody statystyczne - modyfikacja właściwości statystycznych nośnika
	\item metody generacji nośnika - ukrywanie informacji podczas tworzenia samego nośnika
	\item metody rozproszonego widma - ukrycie poprzez rozpraszanie danych
	\item metody zniekształceniowe - wprowadzenie zniekształceń do nośnika i pozyskanie informacji poprzez porównanie nośnika oryginalnego i zniekształconego
\end{itemize}
\section{Przykłady rzeczywistych zastosowań steganografii}
\subsection{Historyczne przykłady użycia steganografii}
Pierwsze wzmianki o użyciu technik steganograficznych można odnaleźć u Herodota w V wieku p.n.e.   Opisuje on sposób tajnego przekazu informacji: tyran Histiajos przetrzymywany przez króla perskiego Dariusza postanowił przesłać informację do swego zięcia Arystagorasa z Miletu, tak aby mogła się ona przedostać mimo pilnujących go strażników. Aby tego dokonać na wygolonej głowie swego niewolnika wytatuował przesłanie. Kiedy niewolnikowi odrosły włosy posłał go z oficjalnym, mało istotnym listem. W starożytnym Egipcie i Chinach stosowano atrament sympatyczny, czyli
zapis wiadomości bezbarwną substancją (np. sok z cytryny, który zyskuje barwę przy podgrzaniu). Ponadto już podczas wojny francusko-pruskiej w 1871 a także 2 Wojny Światowej Niemcy wykorzystywali technikę mikrokropek wklejanych
do tekstu maszynopisu. Na mikrokropkach widoczne były zdjęcia wysokiej rozdzielczości.
\begin{figure}[H]
	\centering
	\includegraphics[width=5cm]{mikrokropkowiec}
	\caption{aparat do wykonywania mikrokropek, skala pomniejszenia ok. 1:300}
\end{figure}
\subsection{Eurion}
\begin{figure}[H]
	\centering
	\includegraphics[width=5cm]{eurion}
	\includegraphics[width=5cm]{euro}
	\includegraphics[width=5cm]{close}
	\caption{EURion, przykładowy układ na banknocie euro, dollarze}
\end{figure}
EURion jak i inne podobne zabiegi stanowią metodę przeciwdziałania fałszerstwom. Na banknotach umieszczane
są zbiory kropek o różnych średnicach i względnych pozycjach (te parametry są sekretem). Kropki te tworzą
fingerprint, który jest wykrywany przez oprogramowanie do skanowania (za pomocą metod detekcji wzorca) i
wszelkie próby kopiowania banknotów są blokowane.  
\subsection{Znaki wodne}
Umieszczanie znaków wodnych w plikach ma na celu zamieszczenie informacji o właścicielu praw autorskich.
Wykorzystujemy różnych metod steganografii w celu zapewnienia:
\begin{itemize}
	\item trudności w usunięciu
	\item odporności na transformacje (\textbf{robustness})
	\item niedostrzegalności (\textbf{perceptibility})
	\item przepustowość
\end{itemize} 
z czego najważniejszą cechą jest odporność na transformacje
\begin{figure}[H]
	\centering
	\includegraphics[width=6cm]{watermark}
	\caption{schemat zamieszczania znaku wodnego}
	\includegraphics[width=4cm]{cap}
	\caption{CAP(Coded Anti Piracy) - przykład znaku wodnego zamieszczanego w filmach do identyfikacji źródła nielegalnych kopii}
\end{figure}
\section{Przegląd technik steganografii}
\subsection{Modyfikacja LSB obrazu}
Jest to klasyczny algorytm steganografii, którego główną wadą jest łatwość w wykryciu/zniszczeniu wiadomości
(np. przez wyzerowanie najmłodszych bitów). Przed niechcianym odczytem wiadomości możemy zapobiec 
poprzez zastosowanie kryptografii. Zasada działania algorytmu jest prosta:
\begin{enumerate}
	\item wybierz, w którym kanale zapisać bity wiadomości (r,g,b, a, może obraz czarnobiały?)
        \item zaszyfruj wiadomość wybranym algorytmem kryptograficznym
	\item zastąp stare wartości najmłodszych bitów określonego kanału obrazu kolejnymi bitami zaszyfrowanej wiadomości
\end{enumerate}
Przy odbieraniu wiadomości wyciągamy daną liczbę bitów ukrytych w pliku oraz deszyfrujemy tak skonstruowany szyfr, co daje nam wiadomość wynikową.
\\
Algorytm skutecznie ukrywa wiadomość w obrazie, ponieważ zamiana najmłodszych bitów pliku nie powoduje widocznej dla człowieka zmiany w jego odbiorze. Przy próbie wykonania
tej samej operacji z MSB może okazać się, że zmiana jest na tyle drastyczna, że wcale nie ukrywa naszych szyfrowanych danych.
\\
Analogiczna metoda jest możliwa na plikach dźwiękowych, tylko tam zmieniamy LSB próbek.
\subsection*{lsb\_hide.py}
lsb\_hide.py to nasza implementacja metody omówionej w poprzednim podpunkcie. \\

Skrypt ukrywa w wartościach RGB odpowiedniej ilości pikseli naszą zaszyfrowaną szyfrem AES w trybie licznikowym wiadomość.
Na wyjściu otrzymujemy wygenerowany dla nas $nonce$ oraz $key$. Argumentami potrzebnymi do wywołania są kolejno: ścieżka do pliku obrazu na którym chcemy ukryć wiadomość,
 ścieżka do pliku obrazu który zostanie zapisany jako wynik działania algorytmu oraz wiadomość którą chcemy ukryć. \\

Przykładowe użycie skryptu
\begin{lstlisting}[language=bash]
python3 lsb_hide.py "your_file_path" "your_result_file_path" "your_message"
\end{lstlisting}

\subsection{Ukrywanie obrazów w spektrogramach}
\begin{figure}[H]
	\centering
	\includegraphics[width=5cm]{samobraz}
	\caption{Obraz zamieniony na dźwięk, bez żadnego "ukrywającego" pliku dźwiękowego, dźwięk
	ten jest odbierany jako szum}
\end{figure}
Inną z technik jest ukrywanie danych w określonych częstotliwościach pliku dźwiękowego.
Jak to działa? Zaczynamy od konwersji obrazu na odcienie szarości.
Następnie korzystamy ze wzoru (IDFT - inverse discrete fourier transform)
\begin{gather*}
	x = \Sigma_{y=0}^{H-1}I[x,y]sin(\frac{2\pi f i}{S})
\end{gather*}
gdzie: \\ 
$H$ to wysokość obrazu, \\
$x$ to próbka odpowiadająca $x$-tej kolumnie obrazu,\\
$I[x,y]$ to jasność piksela o współrzędnych $x,y$, \\
$S$ to częstotliwość próbkowania, \\
$i$ to numer próbki w bloku (odwrotna transformacja Fouriera operuje na blokach stałego rozmiaru,
które łączy w wartość wielomianu w punkcie $x$) \\
rozmiar bloku definiuje szerokość umieszczanego obrazu (w sensie ilości próbek) 
\begin{gather*}
 f = y * \frac{f_{max} - f_{min}}{H} + f_{min} 
\end{gather*}
 $f_{min}$ to częstotliwość, od której zaczyna
się dolna krawędź obrazu, $f_{max}$ to górna krawędź (np. $f_{min} = 20$ Hz, $f_{max} = 20$ kHz).
Chcąc dodać ukryty sygnał do istniejącego pliku dźwiękowego dodajemy sygnały
\begin{gather*}
I' = I + \beta x
\end{gather*}
$\beta$ to współczynnik tłumienia mający na celu ukrycie "brzęczenia" ukrytego obrazu
\begin{figure}[H]
	\centering
	\includegraphics[width=4cm]{gorszajakosc_spektro}
	\includegraphics[width=4cm]{szumy_spektro}
	\caption{Z lewej: przytłumiony sygnał obrazu umieszczony w spektrogramie istniejącego pliku dźwiękowego, z prawej: ilustracja szumów przy braku tłumienia obrazu - szum tworzy widoczne gołym okiem wąsy}
\end{figure}
\subsection*{audio.py}
audio.py to nasza implementacja metody omówionej w poprzednim podpunkcie
\begin{figure}[H]
	\centering
	\includegraphics[width=8cm]{audiopyhelp}
	\caption{wywołanie instrukcji help,  flaga -d to omówiony współczynnik tłumienia, -p to ilość 
	próbek na piksel }
\end{figure}
Przykładowe użycia skryptu
\begin{lstlisting}[language=bash]
python3 audio.py --input plik.png --output plik.wav
python3 audio.py --input plik.png --output plik.wav --carrier carrier.wav
\end{lstlisting}
domyślne wartośni nieobowiązkowych parametrów można konfigurować w pierwszych linijkach skryptu
\subsection{Ukrywanie archiwów w obrazach}
Kolejny bardzo prosty sposób na ukrycie danych w obrazach polega na sklejeniu dwóch plików ze sobą, uzyskując tym samym plik polyglot.  Wymaga to jednak, żeby formaty akceptowały "śmieci" przed nagłówkiem.  Przykładami takich
formatów są pdf, rar, zip.
Metodę tą można łatwo wykryć na przykład za pomocą strings/binwalk. \\\\
Zapisywanie:
\begin{lstlisting}[language=bash]
  $ cat microprocessors.jpg JAVA_slajdy.pdf > image.jpg
\end{lstlisting}
W ten sposób ukrywać możemy zwykłe pliki tekstowe
\begin{lstlisting}[language=bash]
  $ cat microprocessors.jpg haslo.txt > image.jpg
\end{lstlisting}
wówczas 
\begin{lstlisting}[language=bash]
 $ strings image.jpg
	...
 	ri]P
	:9w;
	!`{?
	haslo
\end{lstlisting}
\begin{figure}[H]
	\centering
	\includegraphics[width=8cm]{sklejanie}
	\caption{Przykład pliku "poligloty", który może być jednocześnie interpretowany jako .jpg i .pdf}
\end{figure}
Szczególnym przypadkiem złączonych plików są pliki \textbf{GIFAR} - gif + jar.  Przy nieodpowiednio zabezpieczonej stronie umożliwiającej umieszczanie gifów atakujący może uruchomić 
kod z pliku jar będącego częścią zamieszczonego gifa.  Jary podobnie jak wszystkie formaty bazujące na formacie zip umożliwiają umieszczanie dodatkowych bajtów przed nagłówkiem.

\subsection{Homoglify - Twitter Steganography}
Homoglify to znaki, których kształty mogą być interpretowane na różne sposoby.  Chcąc przy ich użyciu ukryć wiadomość musimy:
\begin{enumerate}
	\item zdefiniować alfabet wiadomości 
	\begin{itemize}
		\item ile bitów na znak? np.  6
		\item jak wygląda alfabet np.   \_abcdefghijklmnopqrstuvwxyz123456789
		\item dla powyższego alfabetu znak a będzie miał kod 000001 a np.  l 001100
	\end{itemize}
	\item ustalić tekst wiadomości, która będzie kontenerem
	\item ustalić tekst ukrytej wiadomości i przekodować go na alfabet
	\item dla każdego znaku kontenera
	\begin{enumerate}
		\item sprawdzamy ile ma homoglifów $h$
		\item liczbę różnych homoglifów danego znaku możemy zakodować na $ceil(log_2 h) = b$ bitach
		\item bierzemy $b$ bitów ukrywanej wiadomości i na ich podstawie wybieramy, który z homoglifów zapiszemy do tekstu wyjściowego
		\item czyli na przykład, jeżeli pierwsza litera kontenera to A, które ma 4 homoglify, to bierzemy 2 bity wiadomości, jeżeli jest to 00 to znaku nie zmieniamy, 01 - wybieramy pierwszy homoglif itd...
	\end{enumerate}
\end{enumerate}
Żeby zdekodować wiadomość musimy znać alfabet i ilość bitów na znak.
\begin{figure}[H]
	\centering
	\includegraphics[width=12cm]{homoglify}
	\caption{przykład wiadomości zakodowanej za pomocą Twitter Steganography}
\end{figure}

\subsection{Chaffing i winnowing}
Chaffing i winnowing to metoda z pogranicza kryptografii jak i steganografii. Technikę tą wymyślił Ron Rivest. Załóżmy, że Alicja chce wysłać wiadomość do Bogdana
oraz wymienili między sobą klucz, który będą wykorzystywali w algorytmie MAC. Ponadto pakiety będą wysyłane bit po bicie, bity będą ponumerowane (żeby zachować ich kolejność).
\begin{enumerate}
	\item Alicja wysyła pakiety wiadomości razem z tagiem MAC tych wiadomości
	\item między pakietami losowo wysyła również zanegowane wartości bitów z losowymi wartościami tagu MAC (chaffing)
	\item Bogdan odbiera pakiety i rozważa tylko te, dla których zgadzia się MAC (winnowing)
\end{enumerate}
Metoda ta gwarantuje
\begin{enumerate}
	\item poufność, podsłuchujący nie wie, który pakiet ma poprawny MAC
	\item autentyczność, ponieważ poprawne pakiety są zabezpieczone za pomocą MAC
\end{enumerate}
\begin{figure}[H]
	\centering
	\includegraphics[width=3cm]{chaffingwinnowing}
	\includegraphics[width=8cm]{chaffingwinnowing2}
	\caption{schemat działania metody i przykład}
\end{figure}


\section{Steganoanaliza}
Steganoanaliza (Steganalysis) jest tym samym dla steganografii, czym kryptoanaliza dla kryptografii - sztuką detekcji ukrytych wiadomości.
\begin{figure}[H]
	\centering
	\includegraphics[width=4cm]{edge_ringing}
	\includegraphics[width=4cm]{noringing}
	\caption{obraz z "edge ringing" i bez - jest to przewidywalne zniekształcenie; proste algorytmy steganografii mogą mieć problem z dobrym 
	odwzorowaniem artefaktów o wysokim prawdopodobieństwie wystąpienia}
\end{figure}
\subsection{Zadania steganoanalizy}
\begin{itemize}
	\item detekcja instnienia kanału steganograficznego (steganoanaliza pasywna)
	\item zniszczenie wiadomości w stegokontenerze
	\item ekstrakcja wiadomości ze stegokontenera
\end{itemize}
\subsection{Podział steganoanalizy}
Podział w zależności od posiadanych informacji
\begin{itemize}
	\item atak skierowany na stegoobiekt  - atakujący ma tylko podejrzany obiekt
	\item atak ze znajomością nośnika - atakujący ma dostęp do czystego nośnika
	\item atak ze znajomością wiadomości
	\item atak z wybranym stegoobiektem - atakujący zna algorytm maskowania
	\item atak z wybraną wiadomością
	\item atak ze znanym stegoobiektem - atakujący zna algorytm, czysty nośnik i stegoobiekt
\end{itemize}
\subsection{Metody wykrywania steganografii}
\begin{enumerate}
	\item Metoda Analizy Statycznej\\
Polega na wykrywaniu nieprawidłowości lub nieoczekiwanych wzorców w danych w ich nośnikach, które mogą wskazywać obecność ukrytych danych. Analiza Statyczna może wykorzystywać przy różnego rodzaju plikach, przykładowo:
        \begin{itemize}
            \item rozkład wartości pikseli
            \item histogramy
            \item rozkład częstotliwości
        \end{itemize}
Można wykorzystać również inne parametry statyczne pliku. Na przykład przy wykrywaniu obrazów może wykorzystywać analizę średnich średnich kolorów czy kontrastu.
        \item Metoda analizy bitów najmniej znaczących\\
        inaczej LSB, która została opisana bardzo dokładnie, razem z przykładem w dalszej części dokumentacji.
        \item Metoda analizy spektralnej\\
Polega na analizie widma częstotliwościowego sygnału lub obrazu w celu wykrycia nieprawidłowości. Podczas analizy spektralnej zastosowana jest na przykład transformacja Fouriera. Transformacja Fouriera jest to rozkład sygnału w dziedzinie czasu na jego składowe częstotliwości. W ten sposób otrzymujemy informacje o występowaniu różnych częstotliwości w sygnale. Kiedy porównamy otrzymany informacje z wzorcami dla danego nośnika danych możemy wykryć nieprawidłowości, które sugerują obecność ukrytych danych.
        \item Metoda analizy entropii\\
Entropia jest miarą nieprzewidywalności lub nieokreśloności, więc analiza entropii polega na ocenie stopnia losowości lub regularności danych w celu wykrycia nieprawidłowości, które będą sugerować ukryte dane.
        \item Metoda analizy różnicowej\\
Tę metodę możemy przedstawić w kilku krokach:
        \begin{itemize}
            \item Posiadanie oryginalnego nośnika danych, niestety tu możemy mieć problem
            \item Porównanie z podejrzanym nośnikiem
            \item Obliczanie różnic między oryginalnym nośnikiem, a podejrzanym (np. na poziomie pikseli w obrazie czy próbek dźwięku)
            \item Analizujemy wzorce i anomalie, wyniki mogą nam pomóc w analizie innych plików już bez oryginalnego.
        \end{itemize}
Ta metoda jest dosyć żmudna i zależy od dokładnej analizy wyników jakie dostajemy.
\end{enumerate}
\subsection{Problemy steganoanalizy}
\begin{itemize}
	\item wiele szyfrów ma taką właściwość, że produkuje szyfrogramy przypominające szum biały (o kompletnie płaskim widmie),
	\item barrage noise - bombardowanie stegokontenerami z losowymi/bezwartościowymi informacjami
\end{itemize}
\subsection{Steganoanaliza LSB}
Prosta metoda wykrycia, czy obraz zawiera zakodowaną wiadomość
\begin{enumerate}
	\item dzielimy piksele na bloki
	\item dla każdego bloku liczymy wartość średnią LSB
\end{enumerate}
Jeżeli w obrazie ukryto \textbf{zaszyfrowaną} wiadomość, to niektóre regiony bloków
powinny mieć średnią ilość jedynek $\approx 0.5$ (ze względu na losowy charakter szyfrów). Możliwe są
oczywiście false positives, gdy piksele w niezmodyfikowanym obrazie mają taką charakterystykę.
\\\\
Bloki podejrzane o zawieranie ukrytej wiadomości dobrze widać na porównaniu wykresu oryginalnego pliku z plikiem zmodyfikowanym.
\begin{figure}[H]
	\centering
	\includegraphics[width=8cm]{original_castle.jpg}
	\includegraphics[width=8cm]{original_hist.png}
	\caption{Przykład obrazu oryginalnego wraz z jego histogramem LSB}
\end{figure}
\begin{figure}[H]
	\centering
	\includegraphics[width=8cm]{modified_castle.png}
	\includegraphics[width=8cm]{mod_hist.png}
	\caption{Przykład obrazu zmodyfikowanego wraz z jego histogramem LSB}
\end{figure}
Przy deszyfrowaniu wiadomości potrzebna jest nam znajomość klucza. Sprawdzamy podejrzany blok tworząc wiadomość z bitów LSB bloku oraz próbujemy zdeszyfrować dane wybranym algorytmem poprzez rozpoczęcie prób deszyfrowania od jednego bajtu do całego bloku dodając po kolei po jednym bajcie więcej bloku. Tym sposobem powinniśmy w którymś momencie działania algorytmu otrzymać poprawną, odszyfrowaną wiadomość.

\subsection*{lsb\_find.py i lsb\_decrypt.py}
lsb\_find.py i lsb\_decrypt.py to nasza implementacja metody omówionej w poprzednim podpunkcie. \\

Skrypt lsb\_find.py przeszukuje plik obrazu pod względem podejrzanych bloków o zawieranie ukrytej wiadomości. Po znalezieniu takiego, otrzymujemy jego koordynaty na zdjęciu.\\
Skrypt lsb\_decrypt.py podejmuje próbę odszyfrowania wiadomości na podstawie koordynatów podejrzanych o zawieranie ukrytej wiadomości.\\

Argumentami potrzebnymi do wywołania skryptu lsb\_find.py są kolejno: ścieżka do pliku obrazu, który podejrzewamy o zawieranie ukrytej wiadomości, wielkość bloku, który będzie
 analizowany pod kątem zawierania ukrytej wiadomości oraz próg odchyłki od prawdopodobieństwa $0.5$ uznawanego za charakterystykę szyfru.\\
Argumentami potrzebnymi do wywołania skryptu lsb\_decrypt.py są kolejno: ścieżka do pliku obrazu, który podejrzewamy o zawieranie ukrytej wiadomości, wielkość bloku, który według nas zawiera ukrytą wiadomość, koordynaty ukrytej wiadomości w obrazie (kolejno: szerekość i wysokość), nonce użyty w zaszyfrowaniu wiadomości oraz klucz użyty w zaszyfrowaniu 
wiadomości.\\

Przykładowe użycie skryptu lsb\_find.py
\begin{lstlisting}[language=bash]
python3 lsb_find.py "your_file_path" block_size threshold
\end{lstlisting}

Przykładowe użycie skryptu lsb\_decrypt.py
\begin{lstlisting}[language=bash]
python3 lsb_find.py "your_file_path" block_size width height nonce key
\end{lstlisting}

\section{Narzędzia steganograficzne}
Wymienione narzędzia mogą zarówno ukrywać jak i wykrywać pliki, motadami jakie wykorzystuje dany program.
\subsection{Outguess}
Jest to oprogramowanie steganograficzne do ukrywania i wykrywania danych w obrazach. 
Metoda działania OutGuess opiera się na podziale obrazu na bloki pikseli, a następnie analizie każdego bloku w celu ukrycia lub wykrycia danych. Podobnie jak wyżej opisana steganografia LSB. W trakcie ukrywania danych przeprowadza analizę statyczną pikseli w blokach obrazu i określa czy blok jest odpowiednim miejscem na ukrycie danych. Jeśli tak OutGuess zamienia najmniej znaczące bity pikseli (LSB) w celu zakodowania ukrytych danych. Taka metoda powoduje, że ukryte dane są odporne na chi-square attack, który opiera się na analizie statystyk pierwszego rzędu.
\subsection{OpenStego}
Narzędzie do steganografii, które umożliwia ukrywanie danych w plikach multimedialnych. Wykorzystuje technikę LSB, czyli modyfikuje bity w sposób by nie zostały wykryte ludzkim okiem. OpenStego umożliwia szyfrowanie danych przed ich ukryciem co daje dodatkowe zabezpieczenie przed ich znalezieniem. Posiada również funkcję odkrycia zaszyfrowanych danych przez osoby do tego uprawnione.
\begin{figure}[H]
	\centering
	\includegraphics[width=8cm]{OpenStego.png}
\end{figure}
\section{Narzędzia steganoanalityczne}
\subsection{strings}
strings to linuxowa komenta wypisująca łańcuchy znaków z danego pliku. Dzięki temu możemy pozyskać informacje 
o metadanych ukrytych np. w plikach wykonawczych
\begin{figure}[H]
	\centering
	\includegraphics[width=8cm]{strings_example}
	\caption{przykład użycia strings na pliku wykonawczym w architekturze rv32}
\end{figure}
\subsection{binwalk}
Metoda działania Binwalk polega na skanowaniu plików binarnych w poszukiwaniu charakterystycznych sygnatur dla różnych typów danych. Przy pomocy bazy jaką posiada może wykrywać dane takie jak pliki, obrazy, podpisów cyfrowych czy kodów wykonawczych.
Program umożliwia także analizę entropii pliku, porównywanie różnic binarnych oraz rozpakować znalezione archiwa rekurencyjne.

\begin{figure}[H]
	\centering
	\includegraphics[width=8cm]{binwalk}
	\caption{przykład użycia binwalk na pliku poliglocie jpg + pdf}
\end{figure}
\subsection{StegExpose}
Narzędzie do wykrywania steganografii w obrazach, które jest wyspecjalizowane w wykrywaniu steganografii LSB w obrazach bezstratnych jakimi są PNG czy BMP.  Steg wywodzi się z metod steganalizy opartych na pikselach takich jak Sample Pairs, RS Analysis, Chi-Square attack czy Primary Sets. Oprócz wykrywania obecności steganografii, StegExpose oferuje również steganoanalizę ilościową (określanie długości ukrytej wiadomości).
\begin{figure}[H]
	\centering
	\includegraphics[width=8cm]{stegexpose}
	\caption{przykład użycia StegExpose}
\end{figure}
\subsection{Sonic Visualizer}
Sonic Visualizer to program do wizualizacji plików dźwiękowych, który może być użyteczny do znajdowania
podejrzanych szumów itp.
\begin{figure}[H]
	\centering
	\includegraphics[width=8cm]{sonic}
	\caption{ilustracja wizualizacji sygnału z podejrzanym szumem}
\end{figure}

\section{Źródła}
\begin{itemize}
	\item \url{https://pl.wikipedia.org/wiki/Steganografia_drukarkowa}
	\item \url{https://royalprice.ru/pl/setting/steganografiya-i-stegoanaliz-obzor-sushchestvuyushchih-programm-i-algoritmov/}
	\item \url{https://www.researchgate.net/figure/The-model-of-steganography-and-steganalysis_fig1_333772050}
	\item \url{http://datagenetics.com/blog/september12015/index.html}
	\item \url{https://holloway.nz/steg/}
	\item \url{https://en.wikipedia.org/wiki/Polyglot_(computing)}
	\item \url{https://github.com/livz/cloacked-pixel}
	\item \url{https://github.com/b3dk7/StegExpose}
	\item \url{https://link.springer.com/chapter/10.1007/11424826_54}
	\item \url{https://github.com/fallais/tweg}
	\item \url{http://datagenetics.com/blog/september12015/index.html}
	\item \url{https://carlmastrangelo.com/blog/gamma-steganography}
	\item \url{https://pl.wikipedia.org/wiki/Mikrokropka}
	\item \url{https://pl.wikipedia.org/wiki/Steganografia}
	\item \url{https://en.wikipedia.org/wiki/Chaffing_and_winnowing}
	\item \url{https://www.researchgate.net/figure/Chaff-and-winnow-based-encryption_fig1_2360410}
	\item \url{https://klinikadanych.pl/artykuly/steganologia-metody-ukrywania-informacji}
         \item \url{https://en.wikipedia.org/wiki/OutGuess}
         \item \url{https://github.com/syvaidya/openstego}
         \item \url{https://www.openstego.com/}
         \item \url{https://github.com/b3dk7/StegExpose}
         \item \url{https://www.youtube.com/watch?v=heprU4URpgc}
         \item \url{https://github.com/ReFirmLabs/binwalk}  
\end{itemize}

\end{document}

