\documentclass{article}

\title{Podstawy steganografii}
\author{Dominik Lau, Sebastian Kutny, Tomasz Lewandowski, Maciej Krzyżanowski}

\usepackage{blindtext}
\usepackage{amsmath}
\usepackage[utf8]{inputenc}
\usepackage[polish]{babel}
\usepackage[T1]{fontenc}
\usepackage{listings}
\usepackage{color}
\usepackage{amssymb}
\usepackage{esvect}
\usepackage{graphicx}

\graphicspath{ {./obrazy/} }

\definecolor{dkgreen}{rgb}{0,0.6,0}
\definecolor{gray}{rgb}{0.5,0.5,0.5}
\definecolor{mauve}{rgb}{0.58,0,0.82}

\lstset{frame=tb,
  language=Python,
  aboveskip=3mm,
  belowskip=3mm,
  showstringspaces=false,
  columns=flexible,
  basicstyle={\small\ttfamily},
  numbers=none,
  numberstyle=\tiny\color{gray},
  keywordstyle=\color{blue},
  commentstyle=\color{dkgreen},
  stringstyle=\color{mauve},
  breaklines=true,
  breakatwhitespace=true,
  tabsize=3
}


\begin{document}

\maketitle
\section{Czym jest steganografia? Do czego służy?}
Steganografia polega na ukrywaniu informacji przez ukrywanie komunikacji w innej formie transmisji danych
np. w obrazkach,  plikach dźwiękowych.  Można zadać sobie pytanie po co stosować steganografię? Jest kilka 
zastosowań:
\begin{itemize}
	\item 
\end{itemize}
\section{Podział steganografii}
\begin{itemize}
	\item steganografia czysta - nie jest stosowany żaden klucz, tekst jawny ukrywamy w pliku, jest to metoda 	Security through obscurity (nie spełnia zasady Kerckhoffsa)
	\item steganografia z kluczem prywatnym - przed komunikacją ustalany jest (np. algorytmem DH) klucz
	steganograficzny wykorzystywany potem w algorytmie, następnie ukrywamy tekst jawny w pliku
	\item steganografia z kluczem publicznym - w pliku ukrywamy szyfrogram zaszyfrowany kluczem publicznym
	odbiorcy
\end{itemize}
\section{Algorytmy steganografii w obrazach}
\subsection{Modyfikacja LSB}
\subsection{Gamma trick}
\section{Algorytmy steganografii w plikach audio}
\section{Stegoanaliza}


\end{document}
